% Don't like 10pt? Try 11pt or 12pt
\documentclass[12pt]{article}

\usepackage[utf8]{inputenc}
\usepackage[T1]{fontenc}
\usepackage{ae}
\usepackage[portuges, brazil]{babel}
\usepackage{graphicx}
\usepackage[hmargin={2cm},vmargin={2cm}]{geometry}
\usepackage{caption}
\usepackage{subcaption}
\usepackage{float}
\usepackage{xcolor}
\usepackage{color}
\usepackage{url}
\usepackage{mathtools}
% Definindo novas cores
\definecolor{green}{rgb}{0,0.5,0}
%\definecolor{blue}{rgb}{0,0.5}
\definecolor{yellow}{rgb}{0.5,0.5,0}
\definecolor{gray}{rgb}{0.5,0.5,0.5}
% Configurando layout para mostrar codigos C++
\usepackage{listings}
\lstset{
  language=C++,
  basicstyle=\ttfamily\small,
  keywordstyle=\color{yellow},
  stringstyle=\color{green},
  commentstyle=\color{gray},
  extendedchars=true,
  showspaces=false,
  showstringspaces=false,
  numbers=none,
  numberstyle=\tiny\color{gray},
  breaklines=true,
  backgroundcolor=\color{white},
  breakautoindent=true,
  captionpos=b,
  xleftmargin=0pt,
  frame=tb,
basicstyle=\footnotesize,
}

\usepackage{fancyhdr}
\pagestyle{fancy}
\usepackage{lastpage}
\lhead{\textsc{IFRN}}
\chead{\textsc{TADS - Programação de Computadores}}
\rhead{\textsc{Aventura de Natal II}}
%\lfoot{Esquerda do Rodapé}
\cfoot{\texttt{Página \thepage}/\pageref{LastPage}}
%\rfoot{Direita do Rodapé}
\renewcommand{\headrulewidth}{1pt}
\renewcommand{\footrulewidth}{1pt}

\newcommand{\veasy}{({\color{green} Muito Fácil})}
\newcommand{\easy}{({\color{yellow} Fácil})}
\newcommand{\normal}{({\color{orange} Normal})}
\newcommand{\inte}{({\color{blue} Intermediário})}
\newcommand{\adv}{({\color{red} Avançado})}

\begin{document}
%=====================================================
%					TITLE
%=====================================================
\thispagestyle{empty}
	\begin{center}
		\begin{figure}
		\centering
		\includegraphics[]{fig/if}
		\end{figure}

		{\large \scshape Tecnologia e Análise em Desenvolvimento de Sistemas}\\[1cm]

		\begin{flushleft}
		 	Disciplina de POO\\ Prof. Demétrios Coutinho - 31/08/2023\\[1cm]
		\end{flushleft}
		
		{- Mini Sistema de Vôos -}\\[1cm]
	\end{center}
	
%=====================================================
%					DOCUMENT
%=====================================================	
	\begin{enumerate}
	\item Crie um programa que leia as notas do primeiro \(N_{1}\) e do segundo \(N_{2}\) bimestre e calcule a média ponderada (MP). O algoritmo deve informar se o aluno está "aprovado", "reprovado" ou "em prova final". O aluno estará aprovado com média maior ou igual a 60 e reprovado com média menor que 20. Segue a expressão da média ponderada:
	\[ MP = \frac{2N_{1} + 3N_{2}}{5} \]
	
	\item  Faça um programa para calcular as raízes da equação de Bhaskara. 
	\[ x_{1} = \frac{-b + \sqrt{\Delta}}{2a} \quad x_{2} = \frac{-b - \sqrt{\Delta}}{2a} \]
	onde
	\[ \Delta = b^{2} - 4ac \]
	O algoritmo deve ler os 3 coeficientes \(a\), \(b\) e \(c\) e deve informar o valor de \(x_{1}\) e \(x_{2}\). Se \(\Delta < 0\), o algoritmo deve informar que é impossível realizar a operação.
	
	\item Em uma eleição sindical, concorreram aos cargos de presidente três candidatos (representados pelas variáveis A, B e C). Durante a apuração dos votos, foram computados votos nulos e em branco, além dos votos válidos para cada candidato. Deve ser criado um programa para fazer a leitura da quantidade de votos válidos para cada candidato, além de ler também a quantidade de votos nulos e em branco. Ao final, o programa deve apresentar o número total de eleitores, considerando votos válidos, nulos e em branco; o percentual correspondente aos votos válidos em relação à quantidade de eleitores; o percentual de votos válidos do candidato A em relação à quantidade de eleitores; o mesmo para os candidatos B e C; o percentual de votos nulos em relação à quantidade de eleitores; e, por último, o percentual de votos em branco em relação à quantidade de eleitores.
	
	\item Faça um programa para ler: a descrição do produto (nome), a quantidade adquirida e o preço unitário. Calcule e escreva o total (total = quantidade adquirida \(\times\) preço unitário), o desconto e o total a pagar (total a pagar = total - desconto), sabendo-se que:
	\begin{itemize}
		\item Se a quantidade \(\leq 5\), o desconto será de 2\%
		\item Se a quantidade \(> 5\) e \(\leq 10\), o desconto será de 3\%
		\item Se a quantidade \(> 10\), o desconto será de 5\%
	\end{itemize}
	
	\item  Faça um programa em Portugol Studio para calcular as raízes da equação de Bhaskara. 
	\[ x_{1} = \frac{-b + \sqrt{\Delta}}{2a} \quad x_{2} = \frac{-b - \sqrt{\Delta}}{2a} \]
	onde
	\[ \Delta = b^{2} - 4ac \]
	O algoritmo deve ler os 3 coeficientes \(a\), \(b\) e \(c\) e informar o valor de \(x_{1}\) e \(x_{2}\). Se \(\Delta < 0\) e \(a = 0\), o programa deve informar que é impossível realizar a operação. Se \(\Delta = 0\), informar que as raízes são reais e iguais. Se \(\Delta > 0\), informar que as raízes são reais e diferentes de zero.
	
	\item Implemente um programa chamado fibonacci01.py que recebe um valor inteiro positivo \(L\) e imprime os termos da série de Fibonacci inferiores a \(L\). A sequência de Fibonacci é definida como tendo os dois primeiros termos iguais a 1 e cada termo seguinte é a soma dos dois termos imediatamente anteriores. Desta forma, se fosse fornecido ao programa uma entrada \(L = 15\), o mesmo deveria produzir a seguinte sequência de termos da série: 1, 1, 2, 3, 5, 8, 13.
	
	\item Desenvolva funções para as seguintes opções:
	\begin{enumerate}
		\item Swap: Troque o conteúdo de duas variáveis passadas por referência.
		\item Ordena3: Receba como parâmetros três números inteiros e um flag ordem. Ordene-os em ordem crescente se ordem for V, ou em ordem decrescente se ordem for F (usando passagem de parâmetro por referência e a função Swap()).
		\item EhPrimo: Verifique se um número recebido como parâmetro é primo.
		\item EhPar: Retorne verdadeiro (V) se um número recebido como parâmetro for par e falso (F) caso contrário.
		\item EhAmigo: Retorne verdadeiro (V) se os dois números recebidos como parâmetros forem amigos e falso (F) caso contrário.
		\item mdc: Retorne o Máximo Divisor Comum de 3 números recebidos como parâmetros.
		\item mmc: Retorne o Mínimo Múltiplo Comum de 3 números recebidos como parâmetros.
		\item Fatorial: Retorne o fatorial do número recebido como parâmetro.
	\end{enumerate}
	
		
	\end{enumerate}
	
	
	
\end{document}

























